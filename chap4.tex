\chapter{主要干扰分析与分工况耦合建模}\label{chap:4}

\section{模型构建思路与工况划分}\label{sec:model_idea}

第一章所建立的线性一体化模型 \eqref{eq:sys_state},是系统控制器设计的标称模型。该模型结构清晰且易于分析,但忽略了各类非线性干扰。如第三章 \ref{chap:3} 的量化分析结果所示,系统指向误差在不同工况下由截然不同的干扰源主导。

因此,为支持高精度控制器的设计,本章采用"标称模型 + 主导干扰"的建模策略。此策略依据第三章的分析结论,针对特定工况,在标称模型基础上选择性地耦合该工况下的主导干扰项,构建面向特定任务的分析模型。该方法既能保证模型在特定工况下的分析精度,又保留了标称模型的简洁性,有助于降低后续控制律的设计复杂度。

本章基于第一章的十五维状态空间模型进行扩展。所有工况模型均采用统一的表征形式:
\begin{equation}
\label{eq:dcm_state_space}
\dot{\bm x} = \bm A \bm x + \bm B \bm u + \bm E_d \bm d(\bm x, t)
\end{equation}
其中状态向量 $\bm x \in \mathbb{R}^{15}$ 由式 \eqref{eq:sys_vectors_revised} 定义,包含方位轴、俯仰轴、快反镜X轴、快反镜Y轴的角度、角速度、电流,以及载机姿态角。标称系统矩阵 $\bm A$ 和输入矩阵 $\bm B$ 由式 \eqref{eq:sys_state_A_B_sub} 定义,为线性时不变矩阵。干扰向量 $\bm d(\bm x, t)$ 表示该工况下的主导干扰,干扰输入矩阵 $\bm E_d$ 将干扰施加到状态方程的相应位置。

\section{工况一:高空静止目标跟踪模型}

根据第三章图 \ref{fig:20} 的分析,在目标静止、5000米以上高空工况下,转轴摩擦是主导干扰。伺服框架长时间工作在零速附近,Stribeck效应和粘滑现象等强非线性特性充分激发。

该工况下的主导干扰向量为:
\begin{equation}
\bm d(\bm x) = 
\begin{bmatrix}
    T_{fA}(\omega_A, z_A) \\
    T_{fE}(\omega_E, z_E)
\end{bmatrix}
\end{equation}
其中 $T_{fA}$ 和 $T_{fE}$ 分别为方位轴和俯仰轴的LuGre摩擦力矩,由式 \eqref{eq:lugre_model} 计算。$\omega_A$ 为方位轴角速度,即状态向量中的第2个分量 $x_2$。$\omega_E$ 为俯仰轴角速度,即状态向量中的第5个分量 $x_5$。$z_A$ 和 $z_E$ 为摩擦模型的内部状态,描述轴向摩擦表面的动态特性。

干扰输入矩阵 $\bm E_d$ 为 $15 \times 2$ 矩阵,将摩擦力矩施加到方位轴和俯仰轴的角加速度状态上。根据式 \eqref{eq:simp_gimbal_A} 和 \eqref{eq:simp_gimbal_E},摩擦力矩作为负力矩进入动力学方程,因此定义为:
\begin{equation}
\label{eq:Ed_friction}
\bm E_d = 
\begin{bmatrix}
    0 & 0 \\
    -1/J_A & 0 \\
    0 & 0 \\
    0 & 0 \\
    0 & -1/J_E \\
    \vdots & \vdots \\
    0 & 0
\end{bmatrix}_{15 \times 2}
\end{equation}
其中 $J_A$ 为方位轴转动惯量,$J_E$ 为俯仰轴转动惯量。矩阵中仅第2行和第5行非零,表明摩擦力矩仅作用于两个伺服轴的角加速度状态。

\section{工况二:低空机动目标跟踪模型}

第三章的量化分析表明,在目标移动、3000米以下低空工况下,伺服框架和快反镜需要进行高动态往复运动,此时快反镜的PZT迟滞效应成为最主要的干扰源之一。

在此工况下,精密跟踪层采用PZT驱动。因此,第\ref{sec:model_idea}节中基于VCM的15维一体化标称模型(式\eqref{eq:dcm_state_space})已不再适用。我们必须构建一个新的标称模型,该模型基于PZT的动态特性。

我们构建一个新的13维状态空间模型。该模型的状态向量 $\bm x \in \mathbb{R}^{13}$,其框架层动态(状态 $x_1$ 至 $x_6$)与载机层动态(状态 $x_{11}$ 至 $x_{13}$)与式\eqref{eq:sys_vectors_revised}保持一致,但快反镜层由VCM的三阶动态(角度、角速度、电流)替换为PZT的二阶动态(角度、角速度)。

新的状态向量定义为:
\begin{equation}
\label{eq:state_vec_pzt}
\bm x = \big[ \eta, \omega_A, i_A, \xi, \omega_E, i_E, \gamma_x, \omega_{Fx}, \gamma_y, \omega_{Fy}, \phi_B, \theta_B, \psi_B \big]^T
\end{equation}

该PZT标称模型基于第1.3.3节的二阶动力学(式\eqref{eq:fsm_pzt_dynamics}),并忽略干扰项 $h$ 和 $p_F$。其快反镜子矩阵 $\bm A_{Fx}$ 和 $\bm A_{Fy}$ 均为二阶形式。

在此PZT标称模型的基础上,迟滞干扰并非一个加性干扰力,而是如第\ref{chap:2}章(式\eqref{eq:bouc_wen_model})所定义的输入非线性映射。因此,工况二的耦合模型应表征为:
\begin{equation}
\label{eq:pzt_coupled_model}
\dot{\bm x} = \bm A_{pzt}\bm x + \bm B_{pzt}\bm u_{eff}(\bm u, \bm h)
\end{equation}
其中 $\bm A_{pzt}$ 和 $\bm B_{pzt}$ 是13维PZT标称模型的系统与输入矩阵。 $\bm u_{eff}$ 是由Bouc-Wen模型(式\eqref{eq:bouc_wen_model})定义的包含了迟滞状态 $\bm h$ 的有效电压输入,它通过输入通道的非线性映射进入系统。

\section{工况三:高空机动目标跟踪模型}

对于目标移动、3000米以上高空工况,第三章图 \ref{fig:20} 显示载机平移旋转引起的未配平力矩成为主导干扰。此工况下目标相对角速度降低,伺服框架与快反镜的动态要求下降,摩擦和迟滞效应减弱。然而载机机动产生的惯性力通过质量偏心产生显著的干扰力矩。

该工况下的主导干扰向量为:
\begin{equation}
\bm d(\bm x, t) = 
\begin{bmatrix}
    T_{uA,z}(\bm{a}_B, \bm{r}_A) \\
    T_{uE,y}(\bm{a}_B, \bm{r}_E)
\end{bmatrix}
\end{equation}
其中 $T_{uA,z}$ 和 $T_{uE,y}$ 分别为作用于方位轴Z轴和俯仰轴Y轴的未配平力矩分量,由式 \eqref{eq:unbalanced_torque_a} 和 \eqref{eq:unbalanced_torque_e} 计算。$\bm{a}_B$ 为载机基座在惯性系下的加速度向量,为时变输入。$\bm{r}_A$ 和 $\bm{r}_E$ 分别为方位框架和俯仰框架质心到各自转轴中心的矢量,表示质量偏心。这些力矩是载机加速度、质量偏心以及系统状态的函数。

干扰输入矩阵 $\bm E_d$ 为 $15 \times 2$ 矩阵,将未配平力矩施加到方位轴和俯仰轴的角加速度状态上。其结构与工况一式 \eqref{eq:Ed_friction} 相同,定义为:
\begin{equation}
\label{eq:Ed_unbalanced}
\bm E_d = 
\begin{bmatrix}
    0 & 0 \\
    1/J_A & 0 \\
    0 & 0 \\
    0 & 0 \\
    0 & 1/J_E \\
    \vdots & \vdots \\
    0 & 0
\end{bmatrix}_{15 \times 2}
\end{equation}
其中 $J_A$ 和 $J_E$ 分别为方位轴和俯仰轴的转动惯量。未配平力矩作为外部力矩源施加,因此矩阵元素为正值。该模型准确反映了此工况下动基座效应的主导地位。

%\input{chapter/chapter5.tex}
% ... existing code ...
